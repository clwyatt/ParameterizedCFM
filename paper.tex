\documentclass[]{article}
\usepackage{amsmath}
\usepackage{amsfonts}
\usepackage{amssymb}
\usepackage{graphicx}
\usepackage{fullpage}
\usepackage{style}
\usepackage{fancybox}
\usepackage{fancyvrb}
\usepackage{color}
\usepackage{parskip}
\usepackage{textcomp}
\usepackage{listings}
\usepackage{attrib}

\lstset{basicstyle=\ttfamily,
upquote=true,
breaklines=true,
postbreak=\raisebox{0ex}[0ex][0ex]{\ensuremath{\hookrightarrow}},
breakatwhitespace=true,
numbers=left
}

\newcommand{\vect}[1]{\boldsymbol{#1}}

\title{Simultaneous Optimization of Cost Function Masks and
  Registration Transforms}

\author{Christopher L. Wyatt}

\begin{document}

\maketitle

\begin{abstract}
  Most registration algorithms are cast as optimization problems. In
  practice registration objectives often require cost function masks
  (CFMs) to exclude regions of topological change in the image
  levelsets caused by anatomical variation or pathology. The CFM must
  be specified a-priori, usually from a manual or semi-manual
  segmentation process. This paper describes and approach to
  parameterization of CFMs using levelsets and thier optimization as
  part of the registration process. TODO: the significance of the
  results.
\end{abstract}

\section{Introduction}
\cite{Crum2003} is a place to start.

\section{Methods}
Define a region of support \(\Omega \subset R^n\) for images \(I_0,
I_1 : \Omega \mapsto R\) as the source (moving) and target (fixed)
image respectively, and \(x \in \Omega\).
\par
Define a transform as \(\Phi : \Omega \mapsto \Omega \) with suitable
properties, e.g. a diffeomorphism.
\par
Define a cost function mask \(\Gamma \subset \Omega\), where the subset
\( \{\Omega - \Gamma \}\) indicates an area of a topology change (a
topological defect).
\par
Define an aggregate registration objective
\begin{equation}
f(\Phi,\Gamma) = \int_\Gamma \|\Phi\|_J dx^n + \lambda \int_\Gamma \|
I_1 - (I_0 \circ \Phi) \|_M dx^n
\end{equation}
where \(\|\bullet\|_J\) and \(\|\bullet\|_M\) are chosen norms for the transform and image residual. 
\par
Let \(\Gamma\) be parameterized by the levelset \(\Gamma \equiv \left\{
  x\in |\Omega | \phi(x,\vect{\theta}) < 0 \right\}\). In other words the
positive portion of the levelset is a segmenttion of the topological defect. Then
\begin{equation}
f(\Phi,\vect{\theta}) = \int H(-\phi(x,\vect{\theta}))\|\Phi\|_J dx^n + \lambda \int H(-\phi(x,\vect{\theta}))\|
I_1 - (I_0 \circ \Phi) \|_M dx^n
\end{equation}
where \(H()\) is the Heaviside function \( H(z) =
\frac{1}{2}(1+\mbox{sign}(z)) \) and \( (1-H(z)) = H(-z) \). The trivial minima of the above functional
is when \( \Gamma = \varnothing \) or \(\phi(x,\vect{\theta}) > 0\)
everywhere, so we add a penalty term measuring the size of \(\{\Omega-\Gamma\}\).
\begin{equation}
f(\Phi,\vect{\theta}) = \int H(-\phi(x,\vect{\theta}))\|\Phi\|_J dx^n 
+ \lambda \int H(-\phi(x,\vect{\theta}))\|I_1 - (I_0 \circ \Phi) \|_M dx^n 
+ \mu \int H(\phi(x,\vect{\theta})) dx^n
\end{equation}

\subsection{Levelset parameterization}
The levelset defining the CFM is parameterized using radial basis
functions from \cite{Aghasi2011}.

\begin{equation}
\vect{\theta} = [\vect{\alpha}, \vect{\beta}, \vect{\chi}]
\end{equation} 

\begin{equation}
\phi(x,[\vect{\alpha}, \vect{\beta}, \vect{\chi}]) = \sum_{j = 1}^{m_0} \alpha_j
\psi(\|\beta_j(x - \vect{\chi_j})\|) 
\end{equation}
where \( \vect{\chi_j} \in R^n\).
\par
We use the \(C^2\) compactly supported RBF
\begin{equation}
\psi_{n}(r) = \left ( \max{\left ( 0,1 - r \right )} \right )^{\left\lfloor
  \frac{n}{2}\right\rfloor + 3}\left ( \left ( \left\lfloor
  \frac{n}{2}\right\rfloor + 3 \right ) r+1 \right ) 
\end{equation}

\subsection{Heaviside Approximation}

\begin{equation}
H_\epsilon(z) = \frac{1}{2} \left (  1 + \frac{2}{\pi}
  \arctan{\frac{\pi z}{\epsilon}} \right )
\end{equation}

\section{Examples}

\subsection{One-dimensional images with translation }
In order to gain some intution about the registration process we begin
with a simple example.
\par
Consider one-dimensional images \(I_0(x)\) and \(I_1(x)\) and a
transform, \(\Phi(x) = x - \tau\) for translation parameter \(\tau\).
\par
Let \(\vect{\theta} = [\vect{\alpha}, \vect{\beta}, \vect{\chi}]\), with three terms in the RBF
summation (i.e. \(m_0 = 3\) ).
\par
Assume the norm for the transform and residual are both the squared two-norm.
Then the registation objective is given by:

\begin{eqnarray*}
f(\tau,\vect{\theta}) = & \int H(-\phi(x,\vect{\theta})) \tau^2 dx \\
& + \lambda \int H(-\phi(x,\vect{\theta}))\left (I_1(x) - (I_0(x-\tau)) \right )^2dx \\
& + \mu \int H(\phi(x,\vect{\theta})) dx
\end{eqnarray*}

The first variation of this objective with respect to the parameters
\(\tau, \vect{\alpha}, \vect{\beta}, \vect{\chi}\) is:

\begin{eqnarray*}
\frac{\partial f}{\partial \tau} = & 2\tau\int H(-\phi(x,\vect{\theta})) dx \\
& + 2\lambda\tau \int H(-\phi(x,\vect{\theta}))\left (I_1(x) - (I_0(x-\tau))
\right )\frac{\partial I_0}{\partial y} dx
\end{eqnarray*}
where \(y = x - \tau\) is the moving frame.
\begin{eqnarray*}
\frac{\partial f}{\partial \alpha_j} = & \tau^2\int -\frac{\partial
  H}{\partial \phi}\frac{\partial \phi}{\partial \alpha_j} dx \\
& + \lambda \int -\left (I_1(x) - (I_0(x-\tau)) \right )^2 \frac{\partial
  H}{\partial \phi}\frac{\partial \phi}{\partial \alpha_j} dx \\
& + \mu \int \frac{\partial
  H}{\partial \phi}\frac{\partial \phi}{\partial \alpha_j} dx
\end{eqnarray*}

\begin{eqnarray*}
\frac{\partial f}{\partial \beta_j} = & \tau^2\int -\frac{\partial
  H}{\partial \phi}\frac{\partial \phi}{\partial \beta_j} dx \\
&+ \lambda \int -\left (I_1(x) - (I_0(x-\tau)) \right )^2 \frac{\partial 
  H}{\partial \phi}\frac{\partial \phi}{\partial \beta_j} dx \\
& + \mu \int \frac{\partial
  H}{\partial \phi}\frac{\partial \phi}{\partial \beta_j} dx
\end{eqnarray*}

\begin{eqnarray*}
\frac{\partial f}{\partial \chi_j} = & \tau^2\int -\frac{\partial
  H}{\partial \phi}\frac{\partial \phi}{\partial \chi_j} dx \\
& + \lambda \int -\left (I_1(x) - (I_0(x-\tau)) \right )^2 \frac{\partial
  H}{\partial \phi}\frac{\partial \phi}{\partial \chi_j} dx \\
& + \mu \int \frac{\partial
  H}{\partial \phi}\frac{\partial \phi}{\partial \chi_j} dx
\end{eqnarray*}

\par
Given the approximation \(H_\epsilon(z)\), its derivative is
\begin{equation}
\frac{\partial H_\epsilon(z)}{\partial z} \equiv \delta_\epsilon(z) = 
\frac{\epsilon}{\epsilon^2 + (\pi z)^2}
\end{equation}
\par
This leaves the partial derivatives of the RBF 
\begin{equation}
\phi(x,\vect{\alpha},\vect{\beta},\vect{\chi}) = \sum_{j}^{3}\alpha_j \psi \left [
  ((\beta_j(x-\chi_j))^2)^{\frac{1}{2}} \right ]
\end{equation}
where
\begin{equation}
\psi[r] = \left ( \max{(0,1-r)} \right )^3(3r+1)
\end{equation}
w.r.t. the parameters \(\vect{\alpha}, \vect{\beta}, \chi\).

\begin{equation}
\frac{\partial \phi}{\partial \alpha_j} = \psi \left [ ((\beta_j(x-\chi_j))^2)^{\frac{1}{2}}\right ]
\end{equation}

\begin{equation}
\frac{\partial \phi}{\partial \beta_j} = \alpha_j \frac{\partial
  \psi}{\partial r} \frac{\partial r}{\partial \beta_j}
\end{equation}

\begin{equation}
\frac{\partial \phi}{\partial \chi_j} = \alpha_j \frac{\partial
  \psi}{\partial r} \frac{\partial r}{\partial \chi_j}
\end{equation}
Continuing in the chain
\begin{equation}
\frac{\partial \psi}{\partial r} = 3\left [ \max{(0,1-r)}^3 -
  \max{(0,1-r)}^2(3r+1) \right ]\\
\end{equation}

\begin{equation}
\frac{\partial r}{\partial \beta_j} = 
\frac{\beta_j(x-\chi_j)^2}{((\beta_j(x-\chi_j))^2)^{\frac{1}{2}}}
\end{equation}
\begin{equation}
\frac{\partial r}{\partial \chi_j} = 
\frac{-\beta_j^2(x-\chi_j)}{((\beta_j(x-\chi_j))^2)^{\frac{1}{2}}}
\end{equation}

\subsubsection{Python Implementation}
First we import what we need
<< d['test1d.py|fn|idio|pycon|pyg|l']['imports'] >>
Then we define some simple digital phantoms for testing
<< d['test1d.py|fn|idio|pycon|pyg|l']['phantom0'] >>
<< d['test1d.py|fn|idio|pycon|pyg|l']['phantom1'] >>
<< d['test1d.py|fn|idio|pycon|pyg|l']['phantom2'] >>
where the smoothing kernel is defined as
<< d['test1d.py|fn|idio|pycon|pyg|l']['kernel'] >>
These are shown in Figure X.

Next we define the function to compute the parameterized levelset
<< d['test1d.py|fn|idio|pycon|pyg|l']['levelset'] >>
using the compactly supported radial basis function of degree one
<< d['test1d.py|fn|idio|pycon|pyg|l']['csrbf'] >>
Then the heaviside function with a default \( \epsilon \)
<< d['test1d.py|fn|idio|pycon|pyg|l']['H'] >>
Finally the objective function is computed, splitting it into three
local functions

We then write a function to run a test using the phantoms with a fixed
translation and the bfgs.

\subsection{Results}

\section{Discussion}

\section*{Appendix}
Put extra details here.

\section*{Acknowledgment}
We would like to thank Foo. Include pertinent grant numbers.

\bibliographystyle{plain}
\bibliography{references}
\end{document}
