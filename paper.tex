\documentclass[]{article}
\usepackage{amsmath}
\usepackage{amsfonts}
\usepackage{amssymb}
\usepackage{graphicx}
\usepackage{fullpage}
\usepackage{style}
\usepackage{fancybox}
\usepackage{fancyvrb}
\usepackage{color}
\usepackage{parskip}
\usepackage{textcomp}
\usepackage{listings}
\usepackage{attrib}

\lstset{basicstyle=\ttfamily,
upquote=true,
breaklines=true,
postbreak=\raisebox{0ex}[0ex][0ex]{\ensuremath{\hookrightarrow}},
breakatwhitespace=true,
numbers=left
}

\title{Simultaneous Optimization of Cost Function Masks and
  Registration Transforms}

\author{Christopher L. Wyatt}

\begin{document}

\maketitle

\begin{abstract}
  Most registration algorithms are cast as optimization problems. In
  practice registration objectives often require cost function masks
  (CFMs) to exclude regions of topological change in the image
  levelsets caused by anatomical variation or pathology. The CFM must
  be specified a-priori, usually from a manual or semi-manual
  segmentation process. This paper describes and approach to
  parameterization of CFMs using levelsets and thier optimization as
  part of the registration process. TODO: the significance of the
  results.
\end{abstract}

\section{Introduction}
\cite{Crum2003} is a place to start.

\section{Methods}
Define images as \(I_0, I_1 : R^n \mapsto R\) as the source (moving) and
target (fixed) image respectively, and \(x \in R^n\).
\par
Define a transform as \(\Phi : R^n \mapsto R^n\) with suitable
properties, e.g. a diffeomorphism.
\par
Define a cost function mask \(\Omega \subset R^n\).
\par
Define an aggregate registration objective
\begin{equation}
f(\Phi,\Omega) = \int_\Omega \|\Phi\|_J dx^n + \lambda \int_\Omega \|
I_1 - (I_0 \circ \Phi) \|_M dx^n
\end{equation}
where \(\|\bullet\|_J\) and \(\|\bullet\|_M\) are chosen norms for the transform and image residual. 
\par
Let \(\Omega\) be parameterized by the levelset \(\Omega \equiv \left\{
  x\in R^n | \phi(x,\theta) < 0 \right\}\). Then
\begin{equation}
f(\Phi,\theta) = \int (1- H(\phi(x,\theta)))\|\Phi\|_J dx^n + \lambda \int (1-H(\phi(x,\theta)))\|
I_1 - (I_0 \circ \Phi) \|_M dx^n
\end{equation}
where \(H()\) is the Heaviside function or a continuous
approximation. The trivial minima of the above functional is when
\( \Omega = \varnothing \) or \(\phi(x,\theta) < 0\) everywhere, so we add a penalty term
measuring the size of \(\Omega\).
\begin{eqnarray*}
f(\Phi,\theta) = & \int H(\phi(x,\theta))\|\Phi\|_J dx^n \\
& + \lambda \int H(\phi(x,\theta))\|I_1 - (I_0 \circ \Phi) \|_M dx^n \\
& + \mu \int H(\phi(x,\theta)) dx^n
\end{eqnarray*}

\subsection{Levelset parameterization}
The levelset defining the CFM is parameterized using radial basis
functions from \cite{Aghasi2011}.

\begin{equation}
\theta = [\alpha, \beta, \chi]
\end{equation} 

\begin{equation}
\phi(x,[\alpha, \beta, \chi]) = \sum_{j = 1}^{m_0} \alpha_j
\psi(\|\beta_j(x - \chi_j)\|) 
\end{equation}

We use the \(C^2\) compactly supported RBF
\begin{equation}
\psi_{n}(r) = \left ( \max{\left ( 0,1 - r \right )} \right )^{\left\lfloor
  \frac{n}{2}\right\rfloor + 2}\left ( \left ( \left\lfloor
  \frac{n}{2}\right\rfloor + 2 \right ) r+1 \right ) 
\end{equation}

\subsection{Heaviside Approximation}

\begin{equation}
H_\epsilon(z) = \frac{1}{2} \left (  1 + \frac{2}{\pi}
  \arctan{\frac{\pi z}{\epsilon}} \right )
\end{equation}

\section{Examples}

\subsection{one-dimensional with translation }
In order to gain some intution about the registration process we begin
with a simple example.
\par
Consider one-dimensional images \(I_0(x)\) and \(I_1(x)\) and a
transform, \(\Phi(x) = x - \tau\) for translation parameter \(\tau\).
\par
Let \(\theta = [\alpha, \beta, \chi]\), i.e. a single term in the RBF
summation.
\par
Assume the norm for the transform and residual are both the squared two-norm.
Then the registation objective is given by:

\begin{eqnarray*}
f(\tau,\theta) = & \int H(\phi(x,\theta)) \tau^2 dx \\
& + \lambda \int H(\phi(x,\theta))\left (I_1(x) - (I_0(x-\tau)) \right )^2dx \\
& + \mu \int H(\phi(x,\theta)) dx
\end{eqnarray*}

The first variation of this objective with respect to the parameters
\(\tau, \alpha, \beta, \chi\) is:

\begin{eqnarray*}
\frac{\partial f}{\partial \tau} = & 2\tau\int H(\phi(x,\theta)) dx \\
& + 2\lambda\tau \int H(\phi(x,\theta))\left (I_1(x) - (I_0(x-\tau))
\right )\frac{\partial I_0}{\partial y} dx
\end{eqnarray*}
where \(y = x - \tau\) is the moving frame.
\begin{eqnarray*}
\frac{\partial f}{\partial \alpha} = & \tau^2\int \frac{\partial
  H}{\partial \phi}\frac{\partial \phi}{\partial \alpha} dx \\
& + \lambda \int \left (I_1(x) - (I_0(x-\tau)) \right )^2 \frac{\partial
  H}{\partial \phi}\frac{\partial \phi}{\partial \alpha} dx \\
& + \mu \int \frac{\partial
  H}{\partial \phi}\frac{\partial \phi}{\partial \alpha} dx
\end{eqnarray*}

\begin{eqnarray*}
\frac{\partial f}{\partial \beta} = & \tau^2\int \frac{\partial
  H}{\partial \phi}\frac{\partial \phi}{\partial \beta} dx \\
& + \lambda \int \left (I_1(x) - (I_0(x-\tau)) \right )^2 \frac{\partial
  H}{\partial \phi}\frac{\partial \phi}{\partial \beta} dx \\
& + \mu \int \frac{\partial
  H}{\partial \phi}\frac{\partial \phi}{\partial \beta} dx
\end{eqnarray*}

\begin{eqnarray*}
\frac{\partial f}{\partial \chi} = & \tau^2\int \frac{\partial
  H}{\partial \phi}\frac{\partial \phi}{\partial \chi} dx \\
& + \lambda \int \left (I_1(x) - (I_0(x-\tau)) \right )^2 \frac{\partial
  H}{\partial \phi}\frac{\partial \phi}{\partial \chi} dx \\
& + \mu \int \frac{\partial
  H}{\partial \phi}\frac{\partial \phi}{\partial \chi} dx
\end{eqnarray*}

\par
Given the approximation \(H_\epsilon(z)\), its derivative is
\begin{equation}
\frac{\partial H_\epsilon(z)}{\partial z} \equiv \delta_\epsilon(z) = 
\frac{\epsilon}{\epsilon^2 + (\pi z)^2}
\end{equation}
\par
This leaves the partial derivatives of the RBF 
\begin{equation}
\phi(x,\alpha,\beta,\chi) = \alpha \psi \left [
  ((\beta(x-\chi))^2)^{\frac{1}{2}} \right ]
\end{equation}
\begin{equation}
\psi[r] = \left ( \max{(0,1-r)} \right )^2(2r-1)
\end{equation}
w.r.t. the parameters \(\alpha, \beta, \chi\).

\begin{equation}
\frac{\partial \phi}{\partial \alpha} = \psi \left [ ((\beta(x-\chi))^2)^{\frac{1}{2}}\right ]
\end{equation}

\begin{equation}
\frac{\partial \phi}{\partial \beta} = \alpha \frac{\partial
  \psi}{\partial r} \frac{\partial r}{\partial \beta}
\end{equation}

\begin{equation}
\frac{\partial \phi}{\partial \chi} = \alpha \frac{\partial
  \psi}{\partial r} \frac{\partial r}{\partial \chi}
\end{equation}
Continuing in the chain
\begin{equation}
\frac{\partial \psi}{\partial r} = \left\{ 
\begin{array}{l@{\quad:\quad}l}
4r^2 - 9r + 4 & (1-r) > 0\\
0 & \mbox{else}
\end{array} \right.
\end{equation}
\begin{equation}
\frac{\partial r}{\partial \beta} = 
\frac{\beta(x-\chi)^2}{((\beta(x-\chi))^2)^{\frac{1}{2}}}
\end{equation}
\begin{equation}
\frac{\partial r}{\partial \chi} = 
\frac{-\beta^2(x-\chi)}{((\beta(x-\chi))^2)^{\frac{1}{2}}}
\end{equation}

\subsection{implementation}
First we import what we need
<< d['test1d.py|fn|idio|pycon|pyg|l']['imports'] >>

\section{Results}

\section{Discussion}

\section*{Appendix}
Put extra details here.

\section*{Acknowledgment}
We would like to thank Foo. Include pertinent grant numbers.

\bibliographystyle{plain}
\bibliography{references}
\end{document}
